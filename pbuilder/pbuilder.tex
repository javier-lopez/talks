%\documentclass[svgnames]{beamer}
\documentclass[dvipsnames]{beamer}

\usepackage[english]{babel}
\usepackage[utf8]{inputenc}
\usepackage{pdfpages}
\usepackage{alltt}
\usepackage{verbatim}
\usepackage{hyperref}

\mode<presentation>
{
  \usetheme{Darmstadt}
  %\usetheme{Antibes}
  \usecolortheme[named=black]{structure}
  \usecolortheme{whale}
  \usecolortheme{default}
  %\setbeamercovered{transparent}
}

%%%%%%%%%%%%%%%%%%%%%%%%%%%%%%%%%%%%%%%%%%%%%%%%%%%%%%%%%%%%
%-------------------------CONTENT---------------------------
%%%%%%%%%%%%%%%%%%%%%%%%%%%%%%%%%%%%%%%%%%%%%%%%%%%%%%%%%%%%
\title{Pbuilder}

\author[Javier López]{Javier López - @javier-lopez}

\institute[Testing Classroom Saucy]
{\url{https://wiki.ubuntu.com/Testing/Activities/Classroom/Saucy}}

\date[2013]
     {28th June - Testing Classroom Saucy}

%require an .ps file ($ convert ubuntu.png ubuntu.ps)
\pgfdeclareimage[height=.5cm]{ubuntu-logo}{imgs/ubuntu}
\pgfdeclareimage[height=3.5cm]{crowd}{imgs/crowd}
\pgfdeclareimage[height=3.5cm]{author}{imgs/author}

%pixl inder font
%pixl express -> effect -> default -> Anne
\pgfdeclareimage[height=1cm]{pbuilder_box}{imgs/pbuilder_box}
\pgfdeclareimage[width=10cm]{pbuilder_1}{imgs/pbuilder_1}
\pgfdeclareimage[width=10cm]{pbuilder_2}{imgs/pbuilder_2}
\pgfdeclareimage[width=10cm]{pbuilder_3}{imgs/pbuilder_3}
\pgfdeclareimage[width=10cm]{pbuilder_4}{imgs/pbuilder_4}
\pgfdeclareimage[width=10cm]{pbuilder_5}{imgs/pbuilder_5}
\pgfdeclareimage[width=10cm]{pbuilder_6}{imgs/pbuilder_6}
\pgfdeclareimage[width=10cm]{pbuilder_7}{imgs/pbuilder_7}
\pgfdeclareimage[width=10cm]{pbuilder_8}{imgs/pbuilder_8}
\pgfdeclareimage[width=10cm]{pbuilder_9}{imgs/pbuilder_9}
\pgfdeclareimage[width=10cm]{pbuilder_10}{imgs/pbuilder_10}
\pgfdeclareimage[width=10cm]{pbuilder_11}{imgs/pbuilder_11}
\pgfdeclareimage[width=10cm]{pbuilder_12}{imgs/pbuilder_12}
\pgfdeclareimage[width=10cm]{pbuilder_13}{imgs/pbuilder_13}
\pgfdeclareimage[width=10cm]{pbuilder_14}{imgs/pbuilder_14}
\pgfdeclareimage[width=10cm]{pbuilder_15}{imgs/pbuilder_15}
\pgfdeclareimage[width=10cm]{pbuilder_16}{imgs/pbuilder_16}
\pgfdeclareimage[width=10cm]{pbuilder_17}{imgs/pbuilder_17}

\logo{\pgfuseimage{ubuntu-logo}}

\AtBeginSection[]
{
  \begin{frame}<beamer>
    \frametitle{Index}
    \tableofcontents[currentsection]
  \end{frame}
}

% If you wish to uncover everything in a step-wise fashion, uncomment
% the following command:
%\beamerdefaultoverlayspecification{<+->}

\begin{document}

\begin{frame}
  \titlepage
\end{frame}

%%%%%%%%%%%%%%%%%%%%%%%%%%%%%%%%%%%%%%%%%%%%%%%%%%%%%%%%%%%%
\section[About]{About}
\begin{frame}
  \frametitle{Pbuilder as a Debian-Ubuntu Building dev system}
  %\begin{center}
        \pgfuseimage{crowd}
        \pgfuseimage{author}
  %\end{center}
  \begin{center}
    {Unichi Uekawa -}
    \url{http://tinyurl.com/pbuilder}
  \end{center}
\end{frame}

\begin{frame}
  \frametitle{Alternatives}
  \begin{itemize}
  \item pbuilder-dist
  \item pbuild
  \item sbuild
  \item cowbuilder
  \item qemubuilder
  \item git-builder
  \item ...
  \end{itemize}
\end{frame}
%%%%%%%%%%%%%%%%%%%%%%%%%%%%%%%%%%%%%%%%
\section[Installation]{Pbuilder setup}

\begin{frame}[fragile]
  \frametitle {apt-get mooo}
  %\begin{center}
  %      \pgfuseimage{pbuilder_box}
  %\end{center}

  \begin{center}
      \pgfuseimage{pbuilder_1}
      {\vspace{3 mm}}
      \pgfuseimage{pbuilder_2}
  \end{center}

\begin{verbatim}
   /.
   /usr
   /usr/bin
   /usr/bin/debuild-pbuilder
   /usr/bin/pdebuild
   ...
\end{verbatim}
  \begin{center}
        Bash wrappers for chroot, debootstrap, dpkg-source, etc
  \end{center}
\end{frame}

\begin{frame}[fragile]
  \frametitle{conf}
  \begin{center}
        \pgfuseimage{pbuilder_3}
  \end{center}
\begin{verbatim}
   export DEBEMAIL="adam@domain.com"
   export DEBFULLNAME="Adam Smith"
   export GPGKEY=BC9C8902
\end{verbatim}
  \begin{center}
      {My full configuration -}
      \url{http://git.io/ONwNiA}
  \end{center}
\end{frame}

\begin{frame}[fragile]
  \frametitle{conf}
  \begin{center}
      \pgfuseimage{pbuilder_4}
  \end{center}
  \begin{verbatim}
   UBUNTU_SUITES=("raring" "quantal" "precise")
   UBUNTU_MIRROR="us.archive.ubuntu.com"
   PATH_PBUILDER="/var/cache/pbuilder"
   NAME="$DIST"
   BASETGZ="$PATH_PBUILDER/$NAME/$NAME-base.tgz"
   HOOKDIR="$HOME/.pbuilder-hooks/"
   ...
  \end{verbatim}
  \begin{center}
      {My full configuration -}
      \url{http://git.io/hluthA}
  \end{center}
\end{frame}

%%%%%%%%%%%%%%%%%%%%%%%%%%%%%%%%%%%%%%%%
\section[Usage]{Pbuilder Usage}

\begin{frame}[fragile]
  \frametitle{usage}
  \begin{center}
        \pgfuseimage{pbuilder_5}
  \end{center}
  \begin{verbatim}
   I: Building the build Environment
   I: extracting base tarball [/var/cache/pbuilder
   /saucy-amd64/saucy-amd64-base.tgz]
   E: failed to find /var/cache/pbuilder/saucy-amd64
   /saucy-amd64-base.tgz, have you done <pbuilder
   create> to create your base tarball yet?
  \end{verbatim}

  \begin{center}
        \pgfuseimage{pbuilder_6}
  \end{center}
\end{frame}

\begin{frame}
  \frametitle{use the source Luke}
  \begin{center}
        \pgfuseimage{pbuilder_7}
        {\vspace{3 mm}}
        \pgfuseimage{pbuilder_8}
        {\vspace{3 mm}}
        \pgfuseimage{pbuilder_9}
  \end{center}
  \begin{center}
      {Download and create a deb source (.dsc).}
      {\vspace{0.5 mm}}
      {Helpul to check if our pbuilder setup works}
  \end{center}
\end{frame}

\begin{frame}[fragile]
  \frametitle{building in a pristine env}
  \begin{center}
        \pgfuseimage{pbuilder_10}
  \end{center}
  \begin{verbatim}
   I: using fakeroot in build.
   I: Current time: Thu Jun 27 11:01:00 CDT 2013
   I: pbuilder-time-stamp: 1372348860
   ...
  \end{verbatim}
  \begin{center}
  Result will be copied by default to: \textbf{/var/cache/pbuilder/results/saucy-amd64/}
  \end{center}
\end{frame}

\begin{frame}[fragile]
  \frametitle{results}
  \begin{center}
        \pgfuseimage{pbuilder_11}
  \end{center}
  \begin{verbatim}
  hello_2.8-3.debian.tar.gz  hello_2.8-3_i386.changes
  hello_2.8.orig.tar.gz hello_2.8-3.dsc
  hello_2.8-3_i386.deb lintian
  \end{verbatim}
  \begin{center}
      \url{hello_2.8-3_i386.deb}
  \end{center}
\end{frame}

\begin{frame}[fragile]
  \frametitle{Pbuilder for testing}
  \begin{center}
        \pgfuseimage{pbuilder_12}
  \end{center}
  \begin{itemize}
  \item Packaging work (pkg bugfixes, backports, etc)
  \item SRU testing
  \item Testing of several ubuntu releases
  \item General sandbox
  \item After exit, it deletes all changes
  \end{itemize}
\end{frame}

%%%%%%%%%%%%%%%%%%%%%%%%%%%%%%%%%%%%%%%%
\section[TT]{Tips 'n Tricks}
\begin{frame}[fragile]
  \frametitle{Update your pbuilder env often | Use aliases}
  \begin{center}
        \pgfuseimage{pbuilder_13}
  \end{center}
  \begin{verbatim}
  0 23 * * * DIST=saucy ARCH=i386 pbuilder update
  0 23 * * * DIST=saucy ARCH=i386 pbuilder update
  \end{verbatim}
  \begin{center}
        \pgfuseimage{pbuilder_14}
  \end{center}
  \begin{verbatim}
  ='sudo DIST=raring ARCH=amd64 pbuilder'
  \end{verbatim}
\end{frame}

\begin{frame}[fragile]
  \frametitle{Speeding up}
  \begin{center}
  {fast dependecy resolution}
  \end{center}
  \begin{center}
        \pgfuseimage{pbuilder_15}
  \end{center}
  \begin{verbatim}
  PBUILDERSATISFYDEPENDSCMD=
  "/usr/lib/pbuilder/pbuilder-satisfydepends-gdebi"
  \end{verbatim}
  \begin{center}
  {tmpfs env, fast io}
  \end{center}
  \begin{center}
        \pgfuseimage{pbuilder_16}
  \end{center}
  \begin{verbatim}
  tmpfs ../saucy-amd64/build/ tmpfs defaults,size=1024M
  tmpfs ../saucy-i386/build/ tmpfs defaults,size=1024M
  \end{verbatim}
\end{frame}

begin{frame}[fragile]
  \frametitle{Speeding up}
  \begin{center}
  {reuse of }
  \end{center}
  \begin{center}
        \pgfuseimage{pbuilder_17}
  \end{center}
  \begin{verbatim}
  PBUILDERSATISFYDEPENDSCMD=
  "/usr/lib/pbuilder/pbuilder-satisfydepends-gdebi"
  \end{verbatim}
  \begin{center}
  {tmpfs env, fast io}
  \end{center}
  \begin{center}
        \pgfuseimage{pbuilder_17}
  \end{center}
  \begin{verbatim}
  tmpfs ../saucy-amd64/build/ tmpfs defaults,size=1024M
  tmpfs ../saucy-i386/build/ tmpfs defaults,size=1024M
  \end{verbatim}
\end{frame}

\begin{frame}
  \frametitle {Help}
  \begin{itemize}
  \item \url{http://tinyurl.com/pbuilder}
  \item \url{https://wiki.ubuntu.com/PbuilderHowto}
  \item \#ubuntu-quality, \#ubuntu-motu freenode.net
  \item nicolas.skaggs@canonical.com
  \item me ;)
  \end{itemize}
\end{frame}

\begin{frame}
  \frametitle {¡Thank you!}
  \begin{center}
      {\huge Questions? o@o?}
  \end{center}
  \begin{center}
      {Javier López - javier-lopez@ubuntu.com}
  \end{center}
  \begin{center}
      {@javier-lopez - \#ubuntu-quality}
  \end{center}
\end{frame}
\end{document}

%%
%% Referencias:
%%
%% http://google.com

%\documentclass[svgnames]{beamer}
\documentclass[dvipsnames]{beamer}

\usepackage[english]{babel}
\usepackage[utf8]{inputenc}
\usepackage{pdfpages}
\usepackage{alltt}
\usepackage{verbatim}
\usepackage{hyperref}

\mode<presentation>
{
  \usetheme{Darmstadt}
  %\usetheme{Antibes}
  \usecolortheme[named=black]{structure}
  \usecolortheme{whale}
  \usecolortheme{default}
  %\setbeamercovered{transparent}
}

%%%%%%%%%%%%%%%%%%%%%%%%%%%%%%%%%%%%%%%%%%%%%%%%%%%%%%%%%%%%
%-------------------------CONTENT---------------------------
%%%%%%%%%%%%%%%%%%%%%%%%%%%%%%%%%%%%%%%%%%%%%%%%%%%%%%%%%%%%
\title{Verifying SRU updates}

\author[Javier López]{Javier López - @javier-lopez}

\institute[Testing Classroom Saucy]
{\url{https://wiki.ubuntu.com/Testing/Activities/Classroom/Saucy}}

\date[2013]
     {24th June - Testing Classroom Saucy}

%require an .ps file ($ convert ubuntu.png ubuntu.ps)
\pgfdeclareimage[height=.5cm]{ubuntu-logo}{imgs/ubuntu}
\pgfdeclareimage[height=5cm]{team}{imgs/team}
\pgfdeclareimage[height=5cm]{community}{imgs/community}

\pgfdeclareimage[height=2cm]{updates}{imgs/updates}
\pgfdeclareimage[height=5cm]{proposed_1}{imgs/proposed_1}
\pgfdeclareimage[height=5cm]{proposed_2}{imgs/proposed_2}
\pgfdeclareimage[height=5cm]{proposed_3}{imgs/proposed_3}
\pgfdeclareimage[height=5cm]{proposed_4}{imgs/proposed_4}
\pgfdeclareimage[height=5cm]{proposed_5}{imgs/proposed_5}
\pgfdeclareimage[height=5cm]{proposed_6}{imgs/proposed_6}
\pgfdeclareimage[height=5cm]{proposed_7}{imgs/proposed_7}
\pgfdeclareimage[height=5cm]{proposed_8}{imgs/proposed_8}

\pgfdeclareimage[height=5cm]{jono_quetzal}{imgs/jono_quetzal}
\pgfdeclareimage[height=2cm]{uas_t}{imgs/uas_t}
\pgfdeclareimage[height=5cm]{uas_1}{imgs/uas_1}
\pgfdeclareimage[height=5cm]{uas_2}{imgs/uas_2}
\pgfdeclareimage[height=5cm]{uas_3}{imgs/uas_3}
\pgfdeclareimage[height=5cm]{uas_4}{imgs/uas_4}

\logo{\pgfuseimage{ubuntu-logo}}

\AtBeginSection[]
{
  \begin{frame}<beamer>
    \frametitle{Index}
    \tableofcontents[currentsection]
  \end{frame}
}

% If you wish to uncover everything in a step-wise fashion, uncomment
% the following command:
%\beamerdefaultoverlayspecification{<+->}

\begin{document}

\begin{frame}
  \titlepage
\end{frame}

%%%%%%%%%%%%%%%%%%%%%%%%%%%%%%%%%%%%%%%%%%%%%%%%%%%%%%%%%%%%
\section[Sponsor]{Canonical}

\begin{frame}
    \frametitle{The Canonical Community Team}
  \begin{center}
        \pgfuseimage{team}
  \end{center}
  \begin{center}
    Canonical style
  \end{center}
\end{frame}

\begin{frame}
  \frametitle{Names}
  \begin{itemize}
  \item Jorge Castro
  \item David Planella
  \item John O'Bacon
  \item Daniel Holbach
  \item Nicholas Skaggs (Our boss)
  \item Michael Hall
  \end{itemize}
\end{frame}

\begin{frame}
  \frametitle{Comunity}
  \begin{center}
        \pgfuseimage{community}
  \end{center}
  \begin{center}
    Woooo!, ubuntu members
  \end{center}
\end{frame}

%%%%%%%%%%%%%%%%%%%%%%%%%%%%%%%%%%%%%%%%
\section[-proposed]{Updates -proposed}

\begin{frame}
  \frametitle {Updates -proposed, AKA SRU}
  %\begin{center}
        %\pgfuseimage{updates}
  %\end{center}

  {\em Ingredients:}
  \begin{itemize}
  \item Any Ubuntu version
  \item Internet connection (the fastest the better)
  \item Launchpad account
  \item Optionally 'pbuilder' | virtualbox
  \end{itemize}

  {\vspace{3 mm}
  \em Preparation time:}
  \begin{itemize}
  \item 30 min
  \end{itemize}
\end{frame}

\begin{frame}
  \frametitle{SRU}
  \begin{center}
        \pgfuseimage{proposed_1}
  \end{center}
  \begin{center}
      \url{http://people.canon.../~ubuntu-archive/pending-sru.html}
  \end{center}
\end{frame}

\begin{frame}
  \frametitle{SRU}
  \begin{center}
      \pgfuseimage{proposed_2}
  \end{center}
  \begin{center}
      «Precise»
  \end{center}
\end{frame}

\begin{frame}
  \frametitle{Procedure}
  \begin{itemize}
  \item Confirm that a bug exists
  \item Download and apply an update
  \item Ensure that the update fixes the problem
  \item Leave feedback
  \item Wait a couple of days to see how an update made its way thanks to you! =)
  \end{itemize}
\end{frame}

\begin{frame}
  \frametitle{SRU}
  \begin{center}
        \pgfuseimage{proposed_3}
  \end{center}
  \begin{center}
      Bug \#988819
  \end{center}
\end{frame}

\begin{frame}
  \frametitle{SRU: Lp}
  \begin{center}
        \pgfuseimage{proposed_4}
  \end{center}
  \begin{center}
      \url{https://bugs.launchpad.net/bugs/988819}
  \end{center}
\end{frame}

\begin{frame}
  \frametitle{SRU: testing}
  \begin{center}
        \pgfuseimage{proposed_5}
  \end{center}
  \begin{center}
      When an update is in -proposed, it needs to be tested before it goes to the -updates repository
  \end{center}
\end{frame}

\begin{frame}
  \frametitle{SRU: ejemplo}
  \begin{itemize}
  \item sudo apt-get -y install apache2 libapache2-modsecurity
  \end{itemize}
  \begin{center}
    Action 'configtest' failed.
    The Apache error log may have more information.
    Your apache2 configuration is broken, so we're not restarting it for you
  \end{center}
\end{frame}

\begin{frame}
  \frametitle{SRU: example}
  \begin{itemize}
    \item sudo apt-get remove --purge apache2 libapache2-modsecurity
    \item sudo apt-get autoremove --purge
    \item sudo vi /etc/apt/sources.list
  \end{itemize}
    deb http://us.archive.ubuntu.com/ubuntu/ precise-proposed main restricted universe multiverse
  \begin{itemize}
    \item sudo apt-get update
    \item sudo apt-get -y install apache2 libapache2-modsecurity
  \end{itemize}
    If the bug dissapears, the update is ready =)
\end{frame}

\begin{frame}
  \frametitle{SRU: feedback}
  \begin{center}
        \pgfuseimage{proposed_6}
  \end{center}
  \begin{center}
      Feedback is welcome even if the test wasn't successful, it's actually more important when an update was wrong
  \end{center}
\end{frame}

\begin{frame}
  \frametitle{SRU: wiki}
  \begin{center}
        \pgfuseimage{proposed_8}
  \end{center}
  \begin{center}
      \url{https://wiki.ubuntu.com/Testing/EnableProposed}
  \end{center}
\end{frame}

\begin{frame}
  \frametitle {Help}
  \begin{itemize}
  \item \#ubuntu-testing en freenode.net
  \item nicolas.skaggs@canonical.com
  \item me ;)
  \end{itemize}
\end{frame}

%%%%%%%%%%%%%%%%%%%%%%%%%%%%%%%%%%%%%%%%
\begin{frame}
  \frametitle {¡Thank you!}
  \begin{center}
      {\huge Questions? o@o?}
  \end{center}
  \begin{center}
      {Javier López - javier-lopez@ubuntu.com}
  \end{center}
  \begin{center}
      {@javier-lopez - \#ubuntu-testing}
  \end{center}
\end{frame}
\end{document}

%%
%% Referencias:
%%
%% http://google.com
